%%%%%%%%%%%%%%%%%%%%%%%%%%%%%%%%%%%%%%%%%%%%%%%%%%%%%%%%%%%%%%%%%%%%%%%%%%%%%%%%
%
% insertionsort: sort lists using a primitive insertionsort
% 
% Author:     Jonathan P. Spratte
% License:    LPPL v1.3c or later
% Copyright:  2019
% File:       insertionsort_shared.tex
%
%   This file contains macros which are identical in every version of
%   insertionsort.
%
%%%%%%%%%%%%%%%%%%%%%%%%%%%%%%%%%%%%%%%%%%%%%%%%%%%%%%%%%%%%%%%%%%%%%%%%%%%%%%%%

% doubled definition guard >>=
\begingroup\expandafter\expandafter\expandafter\endgroup
\expandafter\ifx\csname is@testfont\endcsname\relax
\else
  \expandafter\endinput
\fi
%=<<

\font\is@testfont=cmr10

% messages
\def\is@blankelementerror%>>=
  {%
    % ignore for the time being
  }%=<<

% define some quarks for testing and delimiting
\def\is@qnil{\is@qnil}
\def\is@qmark{\is@qmark}
\def\is@qstop{\is@qstop}

% define some logic helper functions, those are generally faster than their
% \expandafter equivalents
\long\def\is@fi@gobbletwo\fi#1#2{\fi}
\long\def\is@fi@secondofthree\fi#1#2#3{\fi#2}
\long\def\is@fi@thirdofthree\fi#1#2#3{\fi#3}
\long\def\is@fi@gobbleuntilqstop\fi#1\is@qstop{\fi}

% function to get the first value of the clist
\long\def\is@getfirst#1,#2\is@qstop{\unexpanded{#1}}
% function to gobble the first value of a clist, call with trailing ',\is@qnil'
\long\def\is@gobblefirst#1,{\is@gobblefirst@a,}
% function to remove a trailing ',\is@qnil', call it with leading ','
\long\def\is@gobblefirst@a#1,\is@qnil{\is@gobblefirst@b#1}
% remove the leading ',' end the \romannumeral
\def\is@gobblefirst@b,{ }
% function to test whether argument is blank
\long\def\is@ifblank#1%>>=
  {%
    \if\relax\detokenize\expandafter{\@gobble #1.}\relax
      \is@fi@secondofthree
    \fi
    \@secondoftwo
  }%=<<
% nonexpandable test whether TeX understands a number as a dimension; #1 must be
% either a number or a dimension
\protected\long\def\is@ifdimen#1%>>=
  {%
    \begingroup
    \setbox0\hbox{\is@testfont\dimen0=#1pt}%
    \expandafter
    \endgroup
    \ifdim\wd0=\z@
      \is@fi@thirdofthree
    \fi
    \@firstoftwo
  }%=<<
