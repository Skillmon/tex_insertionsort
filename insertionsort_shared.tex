%%%%%%%%%%%%%%%%%%%%%%%%%%%%%%%%%%%%%%%%%%%%%%%%%%%%%%%%%%%%%%%%%%%%%%%%%%%%%%%%
%
% insertionsort: sort lists using a primitive insertionsort
% 
% Author:     Jonathan P. Spratte
% License:    LPPL v1.3c or later
% Copyright:  2019
% File:       insertionsort_shared.tex
%
%   This file contains macros which are identical in every version of
%   insertionsort.
%
%%%%%%%%%%%%%%%%%%%%%%%%%%%%%%%%%%%%%%%%%%%%%%%%%%%%%%%%%%%%%%%%%%%%%%%%%%%%%%%%

% doubled input guard >>=
\csname inso@endinput\endcsname
\expandafter\let\csname inso@endinput\endcsname\endinput
%=<<

\newdimen\inso@unit \inso@unit=1pt
\newdimen\inso@gobbledimen
\newdimen\inso@z@

% messages
\def\inso@blankelementerror%>>=
  {%
    % ignore for the time being
  }%=<<

% define some quarks for testing and delimiting
\def\inso@qnil {\inso@qnil}
\def\inso@qmark{\inso@qmark}
\def\inso@qstop{\inso@qstop}

% define the latex equivalents of some small things
\long\def\inso@gobble#1{}          % don't remove, used in documentation
\long\def\inso@firstofone#1{#1}    % don't remove, used in documentation
\long\def\inso@firstoftwo#1#2{#1}  % don't remove, used in documentation
\long\def\inso@secondoftwo#1#2{#2} % don't remove, used in documentation
\def\inso@empty{}

% define some logic helper functions, those are generally faster than their
% \expandafter equivalents
\long\def\inso@fi@gobble      \fi \inso@firstofone  #1{\fi}
\long\def\inso@fi@firstofone  \fi \inso@gobble      #1{\fi#1}
\long\def\inso@fi@firstoftwo  \fi \inso@secondoftwo #1#2{\fi#1}
\long\def\inso@fi@secondoftwo \fi \inso@firstoftwo  #1#2{\fi#2}

% function to test whether argument is blank
\long\def\inso@ifblank#1%>>=
  {%
    \if\relax\detokenize\expandafter{\inso@gobble #1.}\relax
      \inso@fi@firstoftwo
    \fi
    \inso@secondoftwo
  }%=<<
% helper function for a quick and dirty ifblank test
\long\def\inso@qdifblank#1%>>=
  {%
    \ifx\inso@qstop#1%
  }%=<<
\long\def\inso@ifdimen@branch% evil definition! >>=
  \inso@ifdimen@false#1#2%
  {#1}%=<<
% this will only be expanded if the argument of \inso@ifdimen is no dimen, if so
% provide a unit and rearrange the branches so that the false branch will be
% called by \inso@ifdimen@branch
\def\inso@ifdimen@false{\inso@z@\inso@ifdimen@false\inso@firstofone}
% non-expandable test whether TeX understands something as a dimension; #1 must
% be either a number or a dimension. The arguments to \inso@ifdimen@branch will
% be:
%  1. If the argument to \inso@ifdimen is a dimension:
%     #1=<true>, #2=<false>
%  2. If the argument to \inso@ifdimen is an integer or float:
%     #1=\inso@firstofone, #2=<true>
\protected\long\def\inso@ifdimen#1%>>=
  {%
    \afterassignment\inso@ifdimen@branch % evil definition, watch out!
    \inso@gobbledimen #1 \inso@ifdimen@false
  }%=<<
